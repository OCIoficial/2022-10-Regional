\documentclass{oci}
\usepackage[utf8]{inputenc}
\usepackage{lipsum}

\title{Contraseña}

\begin{document}
\begin{problemDescription}
  Diego es uno de los líderes de la Unión Chilena de Hackers (UCH), lo cual lo pone
  en constante riesgo de ser atacado por otros hackers de organizaciones rivales.
  Es por esto que Diego siempre encripta sus contraseñas para así poder almacenarlas de
  forma segura y protegerse ante cualquier amenaza.

  Diego debe almacenar un total de $N$ contraseñas.
  Cada contraseña corresponde a una cadena de caracteres en el alfabeto inglés
  (sin la \texttt{ñ}).
  Las contraseñas distinguen entre mayúsculas y minúsculas.
  Para almacenarlas, Diego encriptó cada una de las contraseñas usando el mismo
  esquema de encriptación y las guardó en un archivo.

  El esquema de encriptación se basa en intercambiar caracteres de la contraseña
  aplicando una lista de reglas.
  Cada regla consiste en dos caracteres $c_1$ y $c_2$ indicando los caracteres
  que se deben intercambiar.
  De forma concreta, para aplicar una regla, se deben reemplazar todas las ocurrencias
  del carácter $c_1$ por $c_2$ y todas las ocurrencias de $c_2$ por $c_1$.
  Para encriptar una contraseña todas las reglas en la lista se aplican de forma consecutiva
  de arriba hacia abajo.

  Por ejemplo, supón que queremos encriptar la contraseña \texttt{abanico} usando la siguiente
  lista de reglas:
\begin{center}
\texttt{a b} \\
\texttt{b c}
\end{center}
  Comenzamos aplicando la primera regla, es decir, reemplazando todas las ocurrencias de
  \texttt{a} por \texttt{b} y viceversa.
  Después de aplicar la regla obtenemos la cadena \texttt{babnico}.
  Posteriormente, aplicamos la segunda regla y obtenemos \texttt{cacnibo} que es la contraseña
  encriptada final.

  Diego está muy ocupado con sus labores de líder de la UCH y necesita ayuda para escribir un
  programa que le permita desencriptar rápidamente las contraseñas en el archivo.
  De forma concreta, dada la lista de reglas y las $N$ contraseñas encriptadas, el programa
  debe imprimir la lista de contraseñas desencriptadas.
  ¿Podrías ayudarlo?
\end{problemDescription}

\begin{inputDescription}
  La primera línea de la entrada contiene dos enteros $M$ y $N$
  ($0 < M \leq 10000$ y $0 < N \leq 1000$) correspondientes respectivamente a la cantidad
  de reglas y a la cantidad de contraseñas encriptadas.

  A continuación vienen $M$ líneas describiendo la lista de reglas.
  Cada línea contiene 2 caracteres separados por un espacio describiendo una regla.

  Finalmente, siguen $N$ líneas, cada una conteniendo una cadena de texto de largo mayor que cero
  y menor o igual a 100, correspondiente a una contraseña encriptada.

  Se garantiza que todos los caracteres en la entrada (en las contraseñas encriptadas y la lista
  de reglas) serán caracteres del alfabeto inglés en mayúsculas o minúsculas.
\end{inputDescription}

\begin{outputDescription}
  La salida debe contener $N$ líneas, correspondientes a las contraseñas
  originales desencriptadas en el mismo orden en que aparecieron en la entrada.
\end{outputDescription}

\begin{scoreDescription}
  \subtask{??}
  Se probarán varios casos en que $M=1$, es decir, hay solo una regla.
  \subtask{??}
  Se probarán varios casos en que un mismo carácter no aparece en más de una regla.
  \subtask{??}
  Se probarán varios casos sin restricciones adicionales.
\end{scoreDescription}

\begin{sampleDescription}
\sampleIO{sample-1}
\sampleIO{sample-2}
% \sampleIO{sample-3}
\end{sampleDescription}

\end{document}
