\documentclass{oci}
\usepackage[utf8]{inputenc}
\usepackage{lipsum}

\title{Contraseña}

\begin{document}
\begin{problemDescription}
  Diego es un importante miembro de la unión chilena de hackers (UCH), lo que lo pone en un riesgo constante a ser atacado por otros hackers, por lo que siempre encripta sus contraseñas para asegurarse de que estas se mantengan seguras de cualquier posible amenaza.

  Para esto, Diego almacena todas sus N contraseñas en un archivo de su computador, asegurándose de encriptarlas primero.

  Las contraseñas son encriptadas intercambiando algunos de sus carácteres por otros. Los carácteres a intercambiar son determinados por otro archivo, el cual contiene M reglas de intercambio.

  Sin embargo, Diego necesita poder utilizar sus contraseñas, por lo que tendrá que desencriptarlas. Para esto, necesita crear un programa que reciba las M instrucciones y las N contraseñas, y deberá retornar las N contraseñas desencriptadas. ¿Podrás ayudarlo?
\end{problemDescription}

\begin{inputDescription}
  La primera línea de la entrada contiene dos enteros M y N ($0 < M \leq 10000$ y $0 < N \leq 1000$) correspondiente a la cantidad de reglas, y a la cantidad de contraseñas encriptadas respectivamente.

  A continuación, vienen M líneas, cada una conteniendo 2 carácteres separados por un espacio. Estos corresponden a los 2 carácteres a intercambiar.

  Finalmente, siguen N líneas más, cada una conteniendo una contraseña encriptada consistente sólo de carácteres alfanuméricos, de largo menor o igual a 100.
\end{inputDescription}

\begin{outputDescription}
  La salida debe contener N lineas, con cada linea correspondiendo a la contraseña desencriptada correspondiente.
\end{outputDescription}

\begin{scoreDescription}
  \subtask{??}
  Se probarán varios casos en que M=1
  \subtask{??}
  Se probarán varios casos en que un mismo carácter no aparece en más de una regla.
  \subtask{??}
  Se probarán varios casos sin restricciones adicionales.
\end{scoreDescription}

\begin{sampleDescription}
\sampleIO{sample-1}
\sampleIO{sample-2}
\sampleIO{sample-3}
\end{sampleDescription}

\end{document}
