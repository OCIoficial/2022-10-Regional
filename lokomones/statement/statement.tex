\documentclass{oci}
\usepackage[utf8]{inputenc}
\usepackage{booktabs}

\title{El lokómon de Olivier}

\begin{document}
\begin{problemDescription}
La Liga Lokómon es la mayor instancia a la que un entrenador Lokómon puede aspirar.
Para participar, un entrenador debe clasificar teniendo un buen
rendimiento en la pretemporada.
Debido a la gran cantidad de talentosos entrenadores, clasificar a la liga
es una tarea casi imposible.

Marinier es un obstinado entrenador Lokómon que lleva años tratando de clasificar a la liga.
Su participación en la pretemporada fue casi impecable, pero luego de una mala racha
en su última batalla, quedó abajo por solo unos puntos y no logró clasificar.
Ya cansado de quedar fuera de la liga, este año Marinier ha decidido que es tiempo
de dejar de jugar limpio.

Luego de una intensa investigación, Marinier se enteró de que su mayor rival, Olivier, cometió
un error al inscribir uno de sus Lokómones.
A pesar de ser un simple error y haber ganado sus batallas clasificatorias de
forma justa, inscribir mal un Lokómon puede ser motivo de descalificación.
Marinier no perderá la oportunidad de denunciarlo para así dejar un cupo abierto
que el mismo podrá llenar.
Lamentablemente, Marinier no sabe exactamente cuál de los Lokómones fue el que
estuvo mal inscrito.

Para inscribir sus Lokómones, los participantes envían una \emph{lista de inscripción}
detallando para cada Lokómon los valores de cada uno de sus 6 \emph{atributos} asociados.
El valor de un atributo puede corresponder a una cadena de texto o a un entero.
La siguiente tabla detalla cada uno de los 6 atributos, junto con un carácter identificador
del atributo y si corresponden a una cadena de texto o a un entero.
\begin{center}
\begin{tabular}{lcc}
	\toprule
	\textbf{atributo} & \textbf{identificador} & \textbf{tipo de atributo} \\
	\midrule
	{nombre } & \texttt{n} & texto \\
	{tipo   } & \texttt{t} & texto \\
	{ataque } & \texttt{a} & entero \\
	{defensa} & \texttt{d} & entero \\
	{evasión} & \texttt{e} & entero \\
	{vida   } & \texttt{v} & entero\\
	\bottomrule
\end{tabular}
\end{center}

Marinier sabe que \emph{solo una} de las entradas en la lista de inscripción
de Olivier contiene errores.
Adicionalmente, ha logrado recolectar un conjunto de atributos que sabe que el Lokómon
mal inscrito \emph{no tiene}.
Por ejemplo, Marinier sabe que el nombre del Lokómon mal inscrito \emph{no} es \texttt{Castillo}
o que su ataque \emph{no} es \texttt{100}.
Dada la lista de inscripción y el conjunto de atributos que el Lokómon
mal inscrito no tiene, tu tarea es ayudar a Marinier escribiendo un programa que determine
cuál entrada en la lista de inscripción contiene errores o imprimir que es imposible
determinarlo de forma definitiva.
\end{problemDescription}

\begin{inputDescription}
La primera línea de la entrada consiste de dos enteros $X$ y $Y$
($1 \leq X \leq 10^5$, $1 \leq Y \leq 100$).
El entero $X$ corresponde al largo de la lista que recopiló Marinier con los atributos
que el Lokómon mal inscrito no tiene.
El entero $Y$ corresponde al largo de la lista de inscripción.

Las siguientes $X$ líneas contienen la descripción de los atributos que el Lokómon mal
inscrito no tiene.
Cada línea comienza con un carácter identificador de atributo seguido del valor para el atributo,
el que puede ser un entero o una cadena de texto dependiendo del atributo.

Las siguientes $Y$ líneas describen las entradas en la lista de inscripción.
Cada línea contiene los valores para el nombre, el tipo, el ataque, la defensa,
la evasión y la vida (en ese orden) del Lokómon correspondiente.
Se garantiza que todos los Lokómones en la lista tienen un nombre distinto.

Todos los atributos de tipo texto serán siempre no vacíos, tendrán un largo menor o igual que 20
y estarán compuestos por letras minúsculas del alfabeto inglés.
Tos los atributos de tipo entero serán siempre mayores que cero y menores o iguales
que $10^6$.
\end{inputDescription}

\begin{outputDescription}
La salida debe contener el nombre del Lokómon en la entrada que contiene errores.
En caso de no ser posible determinar de forma exacta cuál es la entrada que contiene errores,
tu programa debe imprimir \texttt{IMPOSIBLE}.
\end{outputDescription}

\begin{scoreDescription}
Este problema no contiene subtareas.
Se probarán varios casos de prueba y se entregará puntaje proporcional al número
de casos de prueba correctos siendo 100 el puntaje máximo.
\end{scoreDescription}

\begin{sampleDescription}
\sampleIO{sample-1}
\sampleIO{sample-2}
\end{sampleDescription}

\end{document}
