\documentclass{oci}
\usepackage[utf8]{inputenc}

\title{El lokomón de Olivier}

\begin{document}
\begin{problemDescription}
La Liga Lokomón es la mayor instancia a la que un entrenador Lokomón puede aspirar. Lamentablemente, a lo largo de los años, debido a la gran cantidad de entrenadores que han aparecido en el medio, para disputar la Liga debes tener un buen rendimiento en la OCI (lOkomónes Campeonato Internacional) y clasificar. Marinier es un empedernido entrenador de Lokomones, pero debido a una mala racha ha perdido el cupo para clasificar a la Liga. Pero no todo está perdido, ya que su mayor rival, Olivier, ha inscrito un Lokomón que no era suyo, y a pesar de que Olivier le ha ganado de manera justa, y la inscripción de ese Lokomón no ha significado una gran diferencia, a Marinier no le da vergüenza acusarlo, ya que prefiere clasificar por secretaría que no clasificar. 
 
	A pesar de que Marinier sabe que Olivier ha inscrito mal uno de sus Lokomones, aún no sabe cuál de todos es, pero gracias a un amigo soplón de Olivier conoce algunas cualidades que el Lokomón mal inscrito \textbf{no} posee, y te pide ayuda para encontrarlo. 
 
Un Lokomón tiene 6 atributos asociados, un nombre de largo $k$ $(1 \leq k \leq 20)$ sin espacios, un tipo de largo $j$ $(1 \leq j \leq 20)$ sin espacios, un ataque $a$ $(1 \leq a \leq 10^{6})$, una defensa $d$ $(1 \leq d \leq 10^{6})$, una evasión $e$ $(1 \leq e \leq 10^{6})$ y una vida $v$ $(1 \leq v \leq 10^{6})$. Marinier te da una lista de atributos que sabes que el Lokomón mal inscrito no tiene (por ejemplo, que su nombre no es Castillo, o que su ataque no es 100), y te pide encontrar el nombre del Lokomón que está mal inscrito. Ayúdalo haciendo un programa que lo ayude a encontrarlo, o que le diga que no puede hacerlo en caso de que falte información. 

\end{problemDescription}

\begin{inputDescription}
	La primera línea de la entrada consiste de dos enteros $n$ y $m$ ($1 \leq n \leq 10^5$, $1 \leq m \leq 100$), el largo de la lista que te entrega Marinier y el largo de la lista de lokomones de Olivier.

Las siguientes $n$ líneas contienen un atributo de los lokomones y un valor asociado, el cual sabemos que es distinto al valor del lokomón mal inscrito.

Las siguientes $m$ líneas contienen el nombre, el tipo, el ataque, la defensa, la evasión y la vida (en ese orden) de cada Lokomón de Olivier.
\end{inputDescription}

\begin{outputDescription}
El nombre del Lokomón (si quedan varios lokomones distintos pero con el mismo nombre, si se puede determinar el nombre), en caso de que haya mas de un nombre entre los candidatos, imprimir “NO SE PUEDE SABER”, puedes asumir que como mínimo un lokomon de Olivier cumple con los requisitos para ser el posible lokomón mal inscrito.
\end{outputDescription}

\begin{sampleDescription}
\sampleIO{sample-1}
\sampleIO{sample-2}
\sampleIO{sample-3}
\end{sampleDescription}

\end{document}
