\documentclass{oci}
\usepackage[utf8]{inputenc}
\usepackage{lipsum}

\title{Daniela y los múltiplos}

\begin{document}
\begin{problemDescription}
Daniela es una gran fanática de los números.
Tanto es su fanatismo, que frecuentemente dedica infinito tiempo a
escribir listas con infinitos números en un papel infinito.
La mayoría de los números le fascinan, pero con el paso del tiempo,
Daniela también ha coleccionado una lista (finita) de números que no
le gustan.
Tanta es su aversión por estos números que siempre evita escribirlos en sus
listas infinitas.

Recientemente, Daniela se ha obsesionado con los múltiplos.
Para aprender más sobre ellos, Daniela acostumbra escoger un número de partida y
escribir en orden todos los múltiplos de este.
Como podrías esperar, al escribir los múltiplos, Daniela se salta los números que no le gustan.
Por ejemplo, suponiendo que a Daniela no le gusta el 6 y el 12, al escribir los múltiplos
del 3 se los saltará produciendo la lista $3, 9, 15, 18, 21, \ldots$

Dado un número de partida $a$ y un entero $j$, a Daniela le gustaría determinar
qué número se encuentra en la posición $j$-ésima de la lista que escribe para $a$.
?`Podrías ayudarla?
\end{problemDescription}

\begin{inputDescription}
La primera línea contiene un número $n$ ($1 \leq n \leq 10^5$), indicando la cantidad de números que no le gustan
a Daniela.

La segunda línea contiene $n$ enteros positivos separados por espacios,
indicando los números que no le gustan a Daniela.
Se garantiza que todos los números en esta lista serán distintos.

La última línea contiene dos enteros positivos $a$ y $j$ ($1 \leq a, j \leq 10^4$),
indicando que debes encontrar el $j$-ésimo número en la lista que Daniela
escribe para $a$.
\end{inputDescription}

\begin{outputDescription}
La salida debe contener un entero correspondiente al $j$-ésimo entero en la lista
para $a$.
\end{outputDescription}

\begin{scoreDescription}
Este problema no contiene subtareas.
Se probarán varios casos de prueba y se entregará puntaje proporcional al número
de casos de prueba correctos siendo 100 el puntaje máximo.
\end{scoreDescription}


\begin{sampleDescription}
\sampleIO{sample-1}
\sampleIO{sample-2}
\end{sampleDescription}

\end{document}
