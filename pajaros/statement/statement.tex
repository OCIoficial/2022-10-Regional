\documentclass{oci}
\usepackage[utf8]{inputenc}
\usepackage{lipsum}

\title{Pájaros enojados}

\begin{document}
\begin{problemDescription}
Los pájaros enojados quieren eliminar a todos los cerditos. Los cerdos están posicionados en canastas, donde una
canasta puede tener 1 o 2 cerditos, y cada pájaro puede elegir una sola canasta para atacarla.

Los pájaros pueden ser rojos, lo que significa que pueden eliminar un solo cerdo, o pueden ser amarillos, lo que significa
que pueden eliminar dos cerdos siempre y cuando estén en la misma canasta. De esta forma, un pájaro rojo puede atacar
una canasta pero solo eliminará uno de los cerdos y requerirá que se ataque con otro pájaro rojo o uno amarillo para eliminarla
completamente. Un pájaro amarillo también puede atacar una canasta con un solo cerdo, pero perdería su potencial ya que
podría eliminar una canasta con dos cerdos.

Escribe un programa que diga si es posible que los pájaros, de alguna forma, eliminen a todos los cerditos o no.
\end{problemDescription}

\begin{inputDescription}
	La primera línea consiste en dos números enteros $n$ y $m$ ($1 \leq n, m \leq 5 \cdot 10^5$), indicando la cantidad de pájaros y cerditos,
	respectivamente.

	La segunda línea representa los pájaros. Ésta contiene $n$ números enteros separados por espacios, que pueden ser
	1 o 2. Un 1 indica un pájaro rojo, mientras que un 2 indica uno amarillo.

	La última línea describe las canastas. Ésta contiene $m$ números separados por espacios que también pueden ser 1 o 2,
	indicando la cantidad de cerditos en cada canasta.
\end{inputDescription}

\begin{outputDescription}
	Imprime \texttt{SI} si es posible que los pájaros eliminen a todos los cerditos, y \texttt{NO} si no.
\end{outputDescription}

\begin{sampleDescription}
\sampleIO{sample-1}
\sampleIO{sample-2}
\end{sampleDescription}

\end{document}
