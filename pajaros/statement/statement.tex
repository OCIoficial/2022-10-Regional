\documentclass{oci}
\usepackage[utf8]{inputenc}
\usepackage{lipsum}

\title{Pájaros furiosos}

\begin{document}
\begin{problemDescription}
Los pájaros furiosos deben salvar sus huevos de la amenaza de los
cerditos y están planeando una embestida
para eliminarlos a todos de forma definitiva.

Los cerditos están distribuidos en distintas canastas, donde cada
canasta puede contener 1 o 2 cerditos.
Durante la embestida, cada pájaro puede elegir una sola canasta para
atacar y eliminar a los cerditos en ella.
Los pájaros pueden ser rojos o amarillos.
Al atacar una canasta, un pájaro rojo puede eliminar a lo más un
cerdito de esta.
Si una canasta contiene dos cerditos, al ser atacada por un pájaro rojo
solo se eliminará uno estos y el otro deberá ser eliminado por el ataque de
otro pájaro.
Los pájaros amarillos pueden eliminar hasta dos cerditos.
Un pájaro amarillo puede atacar una canasta con solo un cerdito
(eliminándolo de la canasta) pero se perderá parte de su poder
de ataque.

Tu tarea es ayudar a los pájaros furiosos escribiendo un programa
que determine si es posible elegir el ataque de cada pájaro de forma
que se eliminen todos los cerditos.
\end{problemDescription}

\begin{inputDescription}
	La primera línea consiste en dos números enteros $n$ y $m$ ($1 \leq n, m \leq 5 \cdot 10^5$),
	indicando respectivamente la cantidad de pájaros y cerditos.

	La segunda línea contiene $n$ enteros separados por espacios describiendo
	el color de cada pájaro.
	Un 1 indica un pájaro rojo mientras que un 2 indica uno amarillo.

	La última línea describe las canastas y contiene $m$ enteros separados por espacios.
	Los enteros pueden ser 1 o 2, indicando la cantidad de cerditos en cada canasta.
\end{inputDescription}

\begin{outputDescription}
	La salida debe contener \texttt{SI} en caso de ser posible que los pájaros
	eliminen a todos los cerditos, y \texttt{NO} en caso contrario.
\end{outputDescription}

\begin{sampleDescription}
\sampleIO{sample-1}
\sampleIO{sample-2}
\end{sampleDescription}

\end{document}
